\documentclass[12pt,a4paper]{article}

% ============================================================
% PACKAGES
% ============================================================
\usepackage[utf8]{vietnam}
\usepackage[utf8]{inputenc}
\usepackage[T5]{fontenc}
\usepackage{times}
\usepackage{geometry}
\usepackage{graphicx}
\usepackage{amsmath,amssymb}
\usepackage{algorithm}
\usepackage{algpseudocode}
\usepackage{booktabs}
\usepackage{listings}
\usepackage{xcolor}
\usepackage{hyperref}
\usepackage{float}
\usepackage{caption}
\usepackage{subcaption}
\usepackage{multirow}
\usepackage{array}
\usepackage{tikz}
\usetikzlibrary{shapes,arrows,positioning}

% ============================================================
% CẤU HÌNH
% ============================================================
\geometry{left=3cm, right=2cm, top=2.5cm, bottom=2.5cm}
\hypersetup{
    colorlinks=true,
    linkcolor=blue,
    filecolor=magenta,
    urlcolor=cyan,
    citecolor=blue
}

% Cấu hình hiển thị code Python
\lstset{
    language=Python,
    basicstyle=\ttfamily\small,
    keywordstyle=\color{blue}\bfseries,
    stringstyle=\color{red},
    commentstyle=\color{green!60!black}\itshape,
    numbers=left,
    numberstyle=\tiny\color{gray},
    stepnumber=1,
    numbersep=5pt,
    backgroundcolor=\color{gray!10},
    showspaces=false,
    showstringspaces=false,
    frame=single,
    rulecolor=\color{black!30},
    tabsize=4,
    captionpos=b,
    breaklines=true,
    breakatwhitespace=false,
    escapeinside={(*@}{@*)},
    literate={á}{{\'a}}1 {é}{{\'e}}1 {í}{{\'i}}1 {ó}{{\'o}}1 {ú}{{\'u}}1
             {à}{{\`a}}1 {è}{{\`e}}1 {ì}{{\`i}}1 {ò}{{\`o}}1 {ù}{{\`u}}1
             {ả}{{\h{a}}}1 {ẻ}{{\h{e}}}1 {ỉ}{{\h{i}}}1 {ỏ}{{\h{o}}}1 {ủ}{{\h{u}}}1
             {ã}{{\~a}}1 {ẽ}{{\~e}}1 {ĩ}{{\~i}}1 {õ}{{\~o}}1 {ũ}{{\~u}}1
             {ạ}{{\d{a}}}1 {ẹ}{{\d{e}}}1 {ị}{{\d{i}}}1 {ọ}{{\d{o}}}1 {ụ}{{\d{u}}}1
}

% Đổi tên thuật toán sang tiếng Việt
\floatname{algorithm}{Thuật toán}
\renewcommand{\algorithmicrequire}{\textbf{Đầu vào:}}
\renewcommand{\algorithmicensure}{\textbf{Đầu ra:}}

% ============================================================
% BẮT ĐẦU TÀI LIỆU
% ============================================================
\begin{document}

% ============================================================
% TRANG BÌA
% ============================================================
\begin{titlepage}
    \centering
    \vspace*{1cm}
    
    {\Large \textbf{TRƯỜNG ĐẠI HỌC XÂY DỰNG HÀ NỘI}}\\[0.3cm]
    {\large \textbf{KHOA CÔNG NGHỆ THÔNG TIN}}\\[2cm]
    
    \rule{\linewidth}{0.5mm}\\[0.5cm]
    {\LARGE \textbf{BÁO CÁO BÀI TẬP LỚN}}\\[0.3cm]
    {\Large \textbf{MÔN: KHAI PHÁ DỮ LIỆU}}\\[0.5cm]
    \rule{\linewidth}{0.5mm}\\[1.5cm]
    
    {\Large \textbf{Đề tài:}}\\[0.5cm]
    {\large \textit{Ứng dụng thuật toán Eclat trong phân tích hành vi}}\\
    {\large \textit{người dùng qua dữ liệu clickstream để gợi ý nội dung}}\\[2cm]
    
    \begin{flushleft}
        \hspace{3.5cm}\textbf{Giảng viên hướng dẫn:} Phạm Hồng Phong\\[0.5cm]
        \hspace{3.5cm}\textbf{Nhóm sinh viên thực hiện:}\\[0.3cm]
        \hspace{3.5cm}
        \begin{tabular}{@{}l@{\hspace{2cm}}l@{}}
            1. Nguyễn Việt Anh & MSSV: 0203968 \\
            2. Nguyễn Việt Hùng & MSSV: 0208768 \\
            3. Đỗ Quang Hợp & MSSV: 0208568 \\
        \end{tabular}\\[0.3cm]
        \hspace{3.5cm}\textbf{Lớp:} 68CS2\\
    \end{flushleft}
    
    \vfill
    {\large Hà Nội, tháng 2 năm 2026.}
\end{titlepage}

% ============================================================
% MỤC LỤC
% ============================================================
\tableofcontents
\newpage

% ============================================================
% CHƯƠNG 1: TỔNG QUAN
% ============================================================
\section{Tổng quan đề tài}

\subsection{Đặt vấn đề}

Trong thời đại số hóa hiện nay, các website và ứng dụng web tạo ra lượng dữ liệu khổng lồ về hành vi người dùng (clickstream data). Việc phân tích dữ liệu này giúp:
\begin{itemize}
    \item Hiểu rõ hành vi và sở thích của người dùng
    \item Tối ưu hóa trải nghiệm người dùng (UX)
    \item Đưa ra gợi ý nội dung phù hợp, tăng thời gian ở lại trang
    \item Tăng tỷ lệ chuyển đổi và doanh thu cho doanh nghiệp
\end{itemize}

\textbf{Khai phá luật kết hợp (Association Rule Mining)} là một kỹ thuật quan trọng trong lĩnh vực Khai phá Dữ liệu, cho phép khám phá các mối quan hệ tiềm ẩn giữa các mục trong tập dữ liệu lớn \cite{han2011}.

\subsection{Mục tiêu đề tài}

Đề tài này nhằm:
\begin{enumerate}
    \item Nghiên cứu và triển khai thuật toán \textbf{Eclat} từ đầu (from scratch)
    \item Áp dụng vào phân tích dữ liệu clickstream thực tế từ MSNBC.com
    \item Sinh các luật gợi ý dạng: \textit{"Nếu người dùng xem nội dung A, hãy gợi ý nội dung B"}
    \item Đánh giá hiệu quả các luật bằng các chỉ số: Support, Confidence, Lift
\end{enumerate}

\subsection{Phạm vi nghiên cứu}

\begin{itemize}
    \item \textbf{Dữ liệu}: Bộ dữ liệu MSNBC.com Anonymous Web Data từ UCI Machine Learning Repository \cite{uci_msnbc}
    \item \textbf{Thuật toán}: Eclat (Equivalence Class Transformation) \cite{zaki2000}
    \item \textbf{Ngôn ngữ}: Python 3.x
    \item \textbf{Phương pháp}: Triển khai thủ công, không sử dụng thư viện có sẵn
\end{itemize}

% ============================================================
% CHƯƠNG 2: CƠ SỞ LÝ THUYẾT
% ============================================================
\section{Cơ sở lý thuyết}

\subsection{Khai phá luật kết hợp (Association Rule Mining)}

Khai phá luật kết hợp là quá trình tìm kiếm các mối quan hệ thú vị giữa các mục (items) trong tập dữ liệu giao dịch. Phương pháp này được Agrawal et al. giới thiệu lần đầu năm 1993 \cite{agrawal1993}.

\subsubsection{Các khái niệm cơ bản}

\begin{description}
\item[Định nghĩa 2.1 (Itemset):] Cho $I = \{i_1, i_2, ..., i_n\}$ là tập hợp các mục. Một itemset $X$ là tập con của $I$, ký hiệu $X \subseteq I$. Một k-itemset là itemset chứa đúng $k$ phần tử.

\item[Định nghĩa 2.2 (Transaction Database):] Một cơ sở dữ liệu giao dịch $D$ là tập hợp các giao dịch $T = \{T_1, T_2, ..., T_m\}$, trong đó mỗi giao dịch $T_i \subseteq I$ có một định danh duy nhất TID (Transaction ID).

\item[Định nghĩa 2.3 (Support - Độ hỗ trợ):] Độ hỗ trợ của itemset $X$ được định nghĩa là tỷ lệ các giao dịch chứa $X$ trong $D$:
\begin{equation}
    Support(X) = \frac{|\{T \in D : X \subseteq T\}|}{|D|}
\end{equation}
Ý nghĩa: Support cho biết mức độ phổ biến của một itemset trong toàn bộ cơ sở dữ liệu.

\item[Định nghĩa 2.4 (Frequent Itemset - Tập phổ biến):] Một itemset $X$ được gọi là phổ biến nếu $Support(X) \geq min\_support$, với $min\_support$ là ngưỡng do người dùng định nghĩa.

\item[Định nghĩa 2.5 (Association Rule - Luật kết hợp):] Một luật kết hợp có dạng $X \Rightarrow Y$, trong đó $X, Y \subseteq I$ và $X \cap Y = \emptyset$. Luật này có nghĩa là ``Nếu $X$ xuất hiện thì $Y$ cũng có khả năng xuất hiện''.
\end{description}

\subsubsection{Các chỉ số đánh giá luật}

\begin{description}
\item[1. Confidence (Độ tin cậy):]~\\
Công thức:
\begin{equation}
    Confidence(X \Rightarrow Y) = P(Y|X) = \frac{Support(X \cup Y)}{Support(X)}
\end{equation}
Ý nghĩa: Trong số các giao dịch chứa $X$, có bao nhiêu phần trăm cũng chứa $Y$. Confidence thể hiện độ chắc chắn của luật.

\item[2. Lift (Độ tương quan):]~\\
Công thức:
\begin{equation}
    Lift(X \Rightarrow Y) = \frac{Support(X \cup Y)}{Support(X) \times Support(Y)}
\end{equation}
Ý nghĩa của Lift:
\begin{itemize}
    \item $Lift > 1$: $X$ và $Y$ có tương quan tích cực (xuất hiện cùng nhau nhiều hơn kỳ vọng ngẫu nhiên)
    \item $Lift = 1$: $X$ và $Y$ độc lập thống kê
    \item $Lift < 1$: $X$ và $Y$ có tương quan âm (ít xuất hiện cùng nhau hơn kỳ vọng)
\end{itemize}
\end{description}

\subsubsection{Ví dụ minh họa tổng hợp}

Xét cơ sở dữ liệu giao dịch siêu thị với 5 giao dịch:

\begin{table}[H]
\centering
\caption{Cơ sở dữ liệu giao dịch ví dụ}
\begin{tabular}{|c|l|}
\hline
\textbf{TID} & \textbf{Items} \\
\hline
T1 & Bánh mì, Sữa, Bơ \\
T2 & Bánh mì, Sữa \\
T3 & Sữa, Bơ \\
T4 & Bánh mì, Bơ \\
T5 & Bánh mì, Sữa, Bơ \\
\hline
\end{tabular}
\end{table}

\textbf{Bước 1: Xác định các thông số cơ bản}
\begin{itemize}
    \item Tập items: $I = \{\text{Bánh mì, Sữa, Bơ}\}$
    \item Tổng số giao dịch: $|D| = 5$
\end{itemize}

\textbf{Bước 2: Tính Support cho các itemset}

\begin{table}[H]
\centering
\caption{Support của các itemset}
\begin{tabular}{|l|c|c|l|}
\hline
\textbf{Itemset} & \textbf{Số giao dịch} & \textbf{Support} & \textbf{Xuất hiện trong} \\
\hline
\{Bánh mì\} & 4 & 80\% & T1, T2, T4, T5 \\
\{Sữa\} & 4 & 80\% & T1, T2, T3, T5 \\
\{Bơ\} & 4 & 80\% & T1, T3, T4, T5 \\
\{Bánh mì, Sữa\} & 3 & 60\% & T1, T2, T5 \\
\{Bánh mì, Bơ\} & 3 & 60\% & T1, T4, T5 \\
\{Sữa, Bơ\} & 3 & 60\% & T1, T3, T5 \\
\{Bánh mì, Sữa, Bơ\} & 2 & 40\% & T1, T5 \\
\hline
\end{tabular}
\end{table}

\textbf{Bước 3: Xác định Frequent Itemsets}

Nếu đặt $min\_support = 50\%$, thì các itemset phổ biến là:
\begin{itemize}
    \item 1-itemsets: \{Bánh mì\}, \{Sữa\}, \{Bơ\} (Support = 80\%)
    \item 2-itemsets: \{Bánh mì, Sữa\}, \{Bánh mì, Bơ\}, \{Sữa, Bơ\} (Support = 60\%)
\end{itemize}

\textbf{Bước 4: Sinh và đánh giá luật kết hợp}

Xét luật: $\{\text{Bánh mì}\} \Rightarrow \{\text{Sữa}\}$ (``Nếu khách mua Bánh mì, gợi ý họ mua Sữa'')

\textit{Tính Confidence:}
\begin{align*}
    Confidence(\{\text{Bánh mì}\} \Rightarrow \{\text{Sữa}\}) &= \frac{Support(\{\text{Bánh mì, Sữa}\})}{Support(\{\text{Bánh mì}\})} \\
    &= \frac{60\%}{80\%} = \frac{3}{4} = 75\%
\end{align*}
$\Rightarrow$ Trong 4 giao dịch có Bánh mì, có 3 giao dịch cũng có Sữa (75\%).

\textit{Tính Lift:}
\begin{align*}
    Lift(\{\text{Bánh mì}\} \Rightarrow \{\text{Sữa}\}) &= \frac{Support(\{\text{Bánh mì, Sữa}\})}{Support(\{\text{Bánh mì}\}) \times Support(\{\text{Sữa}\})} \\
    &= \frac{60\%}{80\% \times 80\%} = \frac{0.6}{0.64} = 0.9375
\end{align*}
$\Rightarrow$ Lift $< 1$ cho thấy tương quan âm nhẹ.

\textbf{Bước 5: Tổng kết các chỉ số}

\begin{table}[H]
\centering
\caption{Tổng kết luật \{Bánh mì\} $\Rightarrow$ \{Sữa\}}
\begin{tabular}{|l|c|p{8cm}|}
\hline
\textbf{Chỉ số} & \textbf{Giá trị} & \textbf{Ý nghĩa} \\
\hline
Support & 60\% & 60\% giao dịch có cả Bánh mì và Sữa \\
Confidence & 75\% & 75\% người mua Bánh mì cũng mua Sữa \\
Lift & 0.94 & Tương quan âm nhẹ (xuất hiện cùng nhau ít hơn kỳ vọng) \\
\hline
\end{tabular}
\end{table}

\textbf{Kết luận:} Mặc dù Confidence = 75\% khá cao, nhưng Lift = 0.94 < 1 cho thấy Bánh mì và Sữa không thực sự có mối liên hệ mạnh - chúng đều phổ biến nên hay xuất hiện cùng nhau một cách tự nhiên.


\subsection{Thuật toán Eclat}

\subsubsection{Giới thiệu}

Thuật toán \textbf{Eclat (Equivalence Class Clustering and bottom-up Lattice Traversal)} được Mohammed J. Zaki đề xuất năm 2000 \cite{zaki2000}. Đây là một trong những thuật toán hiệu quả nhất để khai phá tập phổ biến, khắc phục những hạn chế của thuật toán Apriori truyền thống.

\textbf{Ý tưởng chính:} Thay vì quét cơ sở dữ liệu nhiều lần như Apriori, Eclat chuyển đổi dữ liệu sang định dạng dọc (Vertical Format) và sử dụng phép giao tập hợp để tính Support.

\textbf{Đặc điểm nổi bật:}
\begin{enumerate}
    \item Sử dụng \textbf{Vertical Data Format} thay vì Horizontal như Apriori
    \item Chỉ cần \textbf{một lần quét} cơ sở dữ liệu (ở bước tiền xử lý)
    \item Sử dụng \textbf{Depth-First Search (DFS)} để duyệt không gian tìm kiếm
    \item Tính Support bằng \textbf{phép giao TID-Sets} - rất nhanh với phép toán tập hợp
    \item Áp dụng \textbf{nguyên lý Apriori} để cắt tỉa sớm các nhánh không thỏa mãn
\end{enumerate}

\subsubsection{Định dạng dữ liệu dọc (Vertical Data Format)}

\textbf{Định dạng ngang (Horizontal Format):} Mỗi dòng là một giao dịch chứa danh sách các items. Đây là cách lưu trữ truyền thống.


\textbf{Định dạng dọc (Vertical Format):} Mỗi item được lưu kèm danh sách TID (Transaction ID) của các giao dịch chứa item đó. Danh sách này gọi là \textbf{TID-Set} hoặc \textbf{Tidset}.

\textbf{Ví dụ chuyển đổi:}

\textit{Định dạng ngang (Horizontal):}
\begin{center}
\begin{tabular}{|c|l|}
\hline
\textbf{TID} & \textbf{Items} \\
\hline
1 & A, B, C \\
2 & B, C, D \\
3 & A, C, D \\
4 & A, B, D \\
\hline
\end{tabular}
\end{center}

\textit{Định dạng dọc (Vertical):}
\begin{center}
\begin{tabular}{|c|l|}
\hline
\textbf{Item} & \textbf{TID-Set} \\
\hline
A & \{1, 3, 4\} \\
B & \{1, 2, 4\} \\
C & \{1, 2, 3\} \\
D & \{2, 3, 4\} \\
\hline
\end{tabular}
\end{center}

\textbf{Ưu điểm của Vertical Format:}
\begin{enumerate}
    \item \textbf{Tính Support nhanh:} Support của itemset = kích thước của TID-Set giao nhau
    \item \textbf{Không cần quét lại DB:} Mọi thao tác đều trên TID-Sets trong bộ nhớ
    \item \textbf{Phù hợp với DFS:} Dễ dàng mở rộng itemset bằng phép giao
\end{enumerate}

\subsubsection{Nguyên lý hoạt động}

\begin{description}
\item[Tính chất quan trọng:] Với hai itemsets $X$ và $Y$, TID-Set của itemset $X \cup Y$ là:
\begin{equation}
    TID(X \cup Y) = TID(X) \cap TID(Y)
\end{equation}

\item[Công thức tính Support:]~
\begin{equation}
    Support(X \cup Y) = \frac{|TID(X) \cap TID(Y)|}{|D|}
\end{equation}

\item[Nguyên lý Apriori (Downward Closure):] Nếu một itemset không phổ biến, thì tất cả các tập cha (superset) của nó cũng không phổ biến. Eclat sử dụng tính chất này để cắt tỉa sớm.
\end{description}

\subsubsection{Các bước thực hiện thuật toán}

\begin{enumerate}
    \item Chuyển đổi cơ sở dữ liệu từ định dạng ngang sang định dạng dọc (TID-Sets).
    \item Lọc bỏ các 1-itemset có Support < min\_support (áp dụng nguyên lý Apriori).
    \item Sắp xếp các items theo thứ tự (thường theo độ phổ biến giảm dần để tối ưu).
    \item Với mỗi item, tính giao TID-Set với các items khác để tạo 2-itemsets, 3-itemsets,... (đệ quy DFS).
    \item Chỉ giữ lại các itemsets có Support >= min\_support.
\end{enumerate}

\subsubsection{Mã giả thuật toán Eclat}

\begin{algorithm}[H]
\caption{Thuật toán Eclat}
\begin{algorithmic}[1]
\Require Cơ sở dữ liệu $D$, ngưỡng $min\_support$
\Ensure Tập hợp tất cả các Frequent Itemsets $F$

\State $F \gets \emptyset$
\State $min\_support\_count \gets |D| \times min\_support$
\State Chuyển đổi $D$ sang định dạng dọc: $TID\_Dict$
\State Lọc các 1-itemsets có $|TID| < min\_support\_count$
\State Sắp xếp $TID\_Dict$ theo $|TID|$ giảm dần

\Procedure{Eclat\_Recursive}{$Prefix$, $TID\_Subset$, $min\_support\_count$}
    \While{$TID\_Subset \neq \emptyset$}
        \State $(item, tids) \gets TID\_Subset.pop(0)$ \Comment{Lấy phần tử đầu}
        \State $NewItemset \gets Prefix \cup \{item\}$
        \State $F \gets F \cup \{(NewItemset, |tids|)\}$ \Comment{Lưu kết quả}
        
        \State $NewSubset \gets \emptyset$
        \ForAll{$(other\_item, other\_tids) \in TID\_Subset$}
            \State $IntersectTids \gets tids \cap other\_tids$ \Comment{Phép giao}
            \If{$|IntersectTids| \geq min\_support\_count$}
                \State $NewSubset \gets NewSubset \cup \{(other\_item, IntersectTids)\}$
            \EndIf
        \EndFor
        
        \If{$NewSubset \neq \emptyset$}
            \State \Call{Eclat\_Recursive}{$NewItemset$, $NewSubset$, $min\_support\_count$}
        \EndIf
    \EndWhile
\EndProcedure

\State \Call{Eclat\_Recursive}{$\emptyset$, $TID\_Dict$, $min\_support\_count$}
\State \Return $F$
\end{algorithmic}
\end{algorithm}

\subsubsection{Ví dụ minh họa từng bước}

Xét cơ sở dữ liệu với 4 giao dịch và $min\_support = 50\%$ (tức itemset phải xuất hiện trong ít nhất 2 giao dịch):

\begin{table}[H]
\centering
\caption{Dữ liệu đầu vào}
\begin{tabular}{|c|l|}
\hline
\textbf{TID} & \textbf{Items} \\
\hline
1 & A, B, C \\
2 & B, C, D \\
3 & A, C, D \\
4 & A, B, D \\
\hline
\end{tabular}
\end{table}

\textbf{Bước 1: Chuyển sang Vertical Format}

\begin{table}[H]
\centering
\caption{TID-Sets của các 1-itemset}
\begin{tabular}{|c|c|c|c|}
\hline
\textbf{Item} & \textbf{TID-Set} & \textbf{Count} & \textbf{Support} \\
\hline
A & \{1, 3, 4\} & 3 & 75\% \\
B & \{1, 2, 4\} & 3 & 75\% \\
C & \{1, 2, 3\} & 3 & 75\% \\
D & \{2, 3, 4\} & 3 & 75\% \\
\hline
\end{tabular}
\end{table}

Tất cả 1-itemsets đều có Support >= 50\%, nên giữ lại hết.

\textbf{Bước 2: Tính TID-Sets cho 2-itemsets (bằng phép giao)}

\begin{table}[H]
\centering
\caption{Tính toán 2-itemsets}
\begin{tabular}{|l|l|c|c|c|}
\hline
\textbf{Itemset} & \textbf{Phép giao} & \textbf{TID-Set} & \textbf{Count} & \textbf{Phổ biến?} \\
\hline
\{A, B\} & \{1,3,4\} $\cap$ \{1,2,4\} & \{1, 4\} & 2 & Có (50\%) \\
\{A, C\} & \{1,3,4\} $\cap$ \{1,2,3\} & \{1, 3\} & 2 & Có (50\%) \\
\{A, D\} & \{1,3,4\} $\cap$ \{2,3,4\} & \{3, 4\} & 2 & Có (50\%) \\
\{B, C\} & \{1,2,4\} $\cap$ \{1,2,3\} & \{1, 2\} & 2 & Có (50\%) \\
\{B, D\} & \{1,2,4\} $\cap$ \{2,3,4\} & \{2, 4\} & 2 & Có (50\%) \\
\{C, D\} & \{1,2,3\} $\cap$ \{2,3,4\} & \{2, 3\} & 2 & Có (50\%) \\
\hline
\end{tabular}
\end{table}

\textbf{Bước 3: Tính TID-Sets cho 3-itemsets}

\begin{table}[H]
\centering
\caption{Tính toán 3-itemsets}
\begin{tabular}{|l|l|c|c|c|}
\hline
\textbf{Itemset} & \textbf{Phép giao} & \textbf{TID-Set} & \textbf{Count} & \textbf{Phổ biến?} \\
\hline
\{A, B, C\} & \{1,4\} $\cap$ \{1,2,3\} & \{1\} & 1 & Không (25\%) \\
\{A, B, D\} & \{1,4\} $\cap$ \{2,3,4\} & \{4\} & 1 & Không (25\%) \\
\{A, C, D\} & \{1,3\} $\cap$ \{2,3,4\} & \{3\} & 1 & Không (25\%) \\
\{B, C, D\} & \{1,2\} $\cap$ \{2,3,4\} & \{2\} & 1 & Không (25\%) \\
\hline
\end{tabular}
\end{table}

Không có 3-itemset nào đạt ngưỡng 50\%, thuật toán dừng.

\textbf{Bước 4: Tổng kết kết quả}

\begin{table}[H]
\centering
\caption{Tất cả Frequent Itemsets với min\_support = 50\%}
\begin{tabular}{|c|l|c|}
\hline
\textbf{Kích thước} & \textbf{Frequent Itemsets} & \textbf{Số lượng} \\
\hline
1-itemset & \{A\}, \{B\}, \{C\}, \{D\} & 4 \\
2-itemset & \{A,B\}, \{A,C\}, \{A,D\}, \{B,C\}, \{B,D\}, \{C,D\} & 6 \\
\hline
\textbf{Tổng} & & \textbf{10} \\
\hline
\end{tabular}
\end{table}

\subsubsection{Độ phức tạp thuật toán}

\begin{itemize}
    \item \textbf{Time Complexity:} $O(n \times m \times k)$
    \begin{itemize}
        \item $n$: Số lượng giao dịch
        \item $m$: Số lượng items unique
        \item $k$: Độ sâu tối đa của cây tìm kiếm (chiều dài itemset lớn nhất)
    \end{itemize}
    \item \textbf{Space Complexity:} $O(n \times m)$ cho việc lưu trữ TID-Sets
\end{itemize}

\textbf{Lưu ý:} Khi dữ liệu sparse (thưa), Eclat rất hiệu quả vì TID-Sets nhỏ. Khi dữ liệu dense (dày đặc), bộ nhớ có thể là vấn đề.

\subsection{So sánh Eclat với Apriori}

\begin{table}[H]
\centering
\caption{So sánh chi tiết giữa Apriori và Eclat}
\begin{tabular}{|l|p{5cm}|p{5cm}|}
\hline
\textbf{Tiêu chí} & \textbf{Apriori} & \textbf{Eclat} \\
\hline
Định dạng dữ liệu & Horizontal (TID, Items) & Vertical (Item, TID-Set) \\
\hline
Số lần quét DB & Nhiều lần (k lần cho k-itemset) & 1 lần (ở bước tiền xử lý) \\
\hline
Chiến lược duyệt & BFS (theo chiều rộng) & DFS (theo chiều sâu) \\
\hline
Tính Support & Đếm trực tiếp trong DB & Phép giao TID-Sets \\
\hline
Bộ nhớ & Thấp (không lưu TID-Sets) & Cao hơn (lưu TID-Sets) \\
\hline
Tốc độ & Chậm hơn (quét DB nhiều lần) & Nhanh hơn (phép giao nhanh) \\
\hline
Phù hợp với & Dữ liệu dense, ít items & Dữ liệu sparse, nhiều items \\
\hline
\end{tabular}
\end{table}

% ============================================================
% CHƯƠNG 3: DỮ LIỆU VÀ TIỀN XỬ LÝ
% ============================================================
\section{Dữ liệu và tiền xử lý}

\subsection{Giới thiệu bộ dữ liệu}

Bộ dữ liệu \textbf{MSNBC.com Anonymous Web Data} được thu thập từ website msnbc.com vào ngày 28/09/1999 và được lưu trữ tại UCI Machine Learning Repository \cite{uci_msnbc}.

\textbf{Thông tin tổng quan:}
\begin{itemize}
    \item \textbf{Số lượng phiên (sessions):} 989,818
    \item \textbf{Số chuyên mục nội dung:} 17
    \item \textbf{Trung bình số lần xem/phiên:} 5.7
    \item \textbf{Định dạng file:} .seq (sequence)
\end{itemize}

\subsection{Cấu trúc dữ liệu}

File dữ liệu \texttt{msnbc.seq} có cấu trúc:
\begin{itemize}
    \item Các dòng bắt đầu bằng \texttt{\%} là metadata (bỏ qua)
    \item Mỗi dòng dữ liệu đại diện cho một phiên truy cập của người dùng
    \item Các số từ 1-17 đại diện cho ID các chuyên mục
\end{itemize}

\subsection{Bảng ánh xạ chuyên mục}

\begin{table}[H]
\centering
\caption{Ánh xạ ID sang tên chuyên mục}
\begin{tabular}{|c|l|c|l|}
\hline
\textbf{ID} & \textbf{Tên chuyên mục} & \textbf{ID} & \textbf{Tên chuyên mục} \\
\hline
1 & Trang chủ (Frontpage) & 10 & Đời sống (Living) \\
2 & Tin tức (News) & 11 & Kinh doanh (Business) \\
3 & Công nghệ (Tech) & 12 & Thể thao (Sports) \\
4 & Địa phương (Local) & 13 & Tóm tắt (Summary) \\
5 & Ý kiến (Opinion) & 14 & Diễn đàn (BBS) \\
6 & Phát sóng (On-air) & 15 & Du lịch (Travel) \\
7 & Tổng hợp (Misc) & 16 & Tin MSN (MSN-News) \\
8 & Thời tiết (Weather) & 17 & Thể thao MSN (MSN-Sports) \\
9 & Sức khỏe (Health) & & \\
\hline
\end{tabular}
\end{table}

\subsection{Quy trình tiền xử lý}

\begin{enumerate}
    \item \textbf{Đọc file:} Bỏ qua các dòng metadata bắt đầu bằng \texttt{\%}
    \item \textbf{Parse dữ liệu:} Tách các ID số theo khoảng trắng
    \item \textbf{Ánh xạ:} Chuyển ID số sang tên chuyên mục
    \item \textbf{Loại trùng lặp:} Dùng \texttt{set()} để loại các click trùng trong cùng phiên
    \item \textbf{Lọc rỗng:} Loại bỏ các phiên không có dữ liệu hợp lệ
\end{enumerate}

\begin{lstlisting}[caption={Code tiền xử lý dữ liệu},label={lst:preprocess}]
def load_data(filepath=DATA_PATH, limit=None):
    dataset = []
    
    with open(filepath, 'r', encoding='utf-8') as f:
        for line in f:
            if limit and len(dataset) >= limit:
                break
            
            clean_line = line.strip()
            
            # Bo qua dong trong va metadata
            if not clean_line or clean_line.startswith('%'):
                continue
            
            # Tach cac ID so
            item_ids = clean_line.split()
            
            # Chuyen sang set (loai trung lap)
            transaction = set()
            for item_id in item_ids:
                if item_id in CATEGORY_NAMES:
                    transaction.add(CATEGORY_NAMES[item_id])
            
            if transaction:
                dataset.append(transaction)
    
    return dataset
\end{lstlisting}

% ============================================================
% CHƯƠNG 4: TRIỂN KHAI THUẬT TOÁN
% ============================================================
\section{Triển khai thuật toán}

\subsection{Kiến trúc hệ thống}

\begin{figure}[H]
\centering
\begin{tikzpicture}[
    node distance=1.5cm,
    box/.style={rectangle, draw, minimum width=3cm, minimum height=1cm, align=center},
    arrow/.style={->, thick}
]
    \node[box] (data) {data\_loader.py\\Đọc dữ liệu};
    \node[box, right=of data] (eclat) {eclat.py\\Thuật toán Eclat};
    \node[box, right=of eclat] (utils) {utils.py\\Sinh luật gợi ý};
    \node[box, below=of eclat] (main) {main.py\\Điều phối};
    
    \draw[arrow] (data) -- (eclat);
    \draw[arrow] (eclat) -- (utils);
    \draw[arrow] (main) -- (data);
    \draw[arrow] (main) -- (eclat);
    \draw[arrow] (main) -- (utils);
\end{tikzpicture}
\caption{Kiến trúc hệ thống}
\end{figure}

\subsection{Module Eclat (eclat.py)}

\subsubsection{Class Eclat}

\begin{lstlisting}[caption={Cấu trúc class Eclat},label={lst:eclat_class}]
class Eclat:
    def __init__(self, min_support=0.02, min_items=1):
        self.min_support = min_support
        self.min_items = min_items
        self.frequent_itemsets = []
        self._total_wb = 0

    def fit(self, dataset):
        self.frequent_itemsets = []
        self._total_wb = len(dataset)
        
        # 1. Tinh nguong support tuyet doi
        min_sup = self._total_wb * self.min_support
        
        # 2. Chuyen doi sang Vertical Data Format (TID-Sets)
        tid_dict = {}
        for tid, transaction in enumerate(dataset):
            for item in transaction:
                if item not in tid_dict:
                    tid_dict[item] = set()
                tid_dict[item].add(tid)
        
        # 3. Loc 1-itemsets (Pruning som)
        sorted_items = []
        for item, tids in tid_dict.items():
            if len(tids) >= min_sup:
                sorted_items.append((item, tids))
        
        # Sap xep items theo tan suat giam dan
        sorted_items.sort(key=lambda x: len(x[1]), reverse=True)
        
        # 4. Bat dau de quy DFS
        self._eclat_recursive(prefix=[], items=sorted_items, min_sup=min_sup)
        
        return self.frequent_itemsets
\end{lstlisting}

\subsubsection{Hàm đệ quy Eclat}

\begin{lstlisting}[caption={Hàm đệ quy cốt lõi},label={lst:eclat_recursive}]
def _eclat_recursive(self, prefix, items, min_sup):
    while items:
        # Lay item dau tien
        item, tids = items.pop(0)
        new_itemset = prefix + [item]
        
        # Luu ket qua
        if len(new_itemset) >= self.min_items:
            self.frequent_itemsets.append((new_itemset, len(tids)))
        
        # Tao danh sach ung vien tiep theo (Intersection)
        next_items = []
        for other_item, other_tids in items:
            new_tids = tids & other_tids  # Phep giao
            
            if len(new_tids) >= min_sup: # Cat tia
                next_items.append((other_item, new_tids))
        
        # De quy DFS
        if next_items:
            self._eclat_recursive(new_itemset, next_items, min_sup)
\end{lstlisting}

\subsection{Module sinh luật (utils.py)}

\begin{lstlisting}[caption={Hàm sinh luật gợi ý},label={lst:generate_rules}]
def generate_recommendation_rules(frequent_itemsets, total_transactions, 
                                  min_confidence=0.5):
    rules = []
    support_lookup = {tuple(sorted(items)): count 
                      for items, count in frequent_itemsets}
    
    import itertools
    for itemset, support_count_AB in frequent_itemsets:
        if len(itemset) < 2: continue
        support_AB = support_count_AB / total_transactions
        
        # Sinh tat ca tap con lam ve trai (X -> Y)
        all_antecedents = []
        for r in range(1, len(itemset)):
            all_antecedents.extend(itertools.combinations(itemset, r))
            
        for antecedent_tuple in all_antecedents:
            antecedent = list(antecedent_tuple)
            consequent = [x for x in itemset if x != antecedent_item]
            
            # Tinh Confidence va Lift
            support_A = support_lookup.get(tuple(antecedent), 0) / total_transactions
            support_B = support_lookup.get(tuple(sorted(consequent)), 0) / total_transactions
            
            confidence = support_AB / support_A if support_A > 0 else 0
            lift = support_AB / (support_A * support_B) if support_A * support_B > 0 else 0
            
            if confidence >= min_confidence:
                rules.append({
                    'antecedent': antecedent,
                    'consequent': consequent,
                    'support': support_AB,
                    'confidence': confidence,
                    'lift': lift
                })
    
    rules.sort(key=lambda x: (x['lift'], x['confidence']), reverse=True)
    return rules
\end{lstlisting}

% ============================================================
% CHƯƠNG 5: KẾT QUẢ VÀ ĐÁNH GIÁ
% ============================================================
\section{Kết quả và đánh giá}

\subsection{Cấu hình thực nghiệm}

\begin{itemize}
    \item \textbf{Số phiên xử lý:} 989,818 (toàn bộ dữ liệu)
    \item \textbf{Min Support:} 0.02 (2\%)
    \item \textbf{Min Confidence:} 0.40 (40\%)
\end{itemize}

\subsection{Kết quả khai phá tập phổ biến}

\begin{itemize}
    \item Số 1-itemsets phổ biến: \textbf{15/17} chuyên mục
    \item Số 2-itemsets phổ biến: \textbf{17} cặp
    \item Tổng tập phổ biến: \textbf{32}
\end{itemize}

\textbf{Top 5 chuyên mục phổ biến nhất:}
\begin{enumerate}
    \item Trang chủ (Frontpage): 940,469 lượt xem
    \item Tin tức (News): 452,387 lượt xem
    \item Thời tiết (Weather): 439,398 lượt xem
    \item Phát sóng (On-air): 414,928 lượt xem
    \item Diễn đàn (BBS): 395,880 lượt xem
\end{enumerate}

\subsection{Kết quả sinh luật gợi ý}

\begin{table}[H]
\centering
\caption{Top 5 luật gợi ý nội dung mạnh nhất}
\begin{tabular}{|c|l|l|c|c|c|}
\hline
\textbf{STT} & \textbf{Nếu xem} & \textbf{Gợi ý} & \textbf{Support} & \textbf{Conf.} & \textbf{Lift} \\
\hline
1 & Tổng hợp & Phát sóng & 3.36\% & 41.3\% & \textbf{1.88} \\
2 & Kinh doanh & Trang chủ & 3.31\% & 56.8\% & 1.80 \\
3 & Đời sống & Trang chủ & 2.65\% & 51.9\% & 1.64 \\
4 & Tổng hợp & Trang chủ & 3.68\% & 45.2\% & 1.43 \\
5 & Tin tức & Trang chủ & 7.55\% & 42.6\% & 1.35 \\
\hline
\end{tabular}
\end{table}

\subsection{Phân tích kết quả}

\begin{description}
    \item[Luật 1: Tổng hợp $\Rightarrow$ Phát sóng (Lift = 1.88)] \hfill
    \begin{itemize}
        \item \textbf{Ý nghĩa}: Người xem mục Tổng hợp có xu hướng xem tiếp mục Phát sóng
        \item \textbf{Xác suất}: Lift = 1.88 cho thấy xác suất cao hơn ngẫu nhiên \textbf{88\%}
        \item \textbf{Đánh giá}: Đây là luật có độ tương quan mạnh nhất
    \end{itemize}

    \item[Luật 2: Kinh doanh $\Rightarrow$ Trang chủ (Conf = 56.8\%)] \hfill
    \begin{itemize}
        \item \textbf{Ý nghĩa}: Hơn 56\% người xem Kinh doanh cũng xem Trang chủ
        \item \textbf{Đánh giá}: Đây là luật có Confidence cao nhất
        \item \textbf{Gợi ý}: Nên đặt banner Trang chủ trong mục Kinh doanh
    \end{itemize}
\end{description}

\subsection{Ứng dụng thực tế}

Dựa trên kết quả phân tích, các gợi ý cho hệ thống:
\begin{enumerate}
    \item Hiển thị widget "Bạn có thể quan tâm" chứa nội dung Phát sóng khi người dùng đang xem Tổng hợp
    \item Thêm link nhanh về Trang chủ trong các mục Kinh doanh, Đời sống, Tin tức
    \item Tối ưu layout để tăng cross-navigation giữa các chuyên mục có Lift cao
\end{enumerate}

% ============================================================
% CHƯƠNG 6: KẾT LUẬN
% ============================================================
\section{Kết luận}

\subsection{Tổng kết}

Đề tài đã hoàn thành các mục tiêu đề ra:
\begin{itemize}
    \item[$\checkmark$] Nghiên cứu và triển khai thành công thuật toán Eclat từ đầu
    \item[$\checkmark$] Xử lý được gần 1 triệu phiên clickstream thực tế
    \item[$\checkmark$] Sinh được các luật gợi ý có ý nghĩa thống kê (Lift > 1)
    \item[$\checkmark$] Đánh giá hiệu quả bằng các chỉ số Support, Confidence, Lift
\end{itemize}

\subsection{Hạn chế}

\begin{itemize}
    \item Dữ liệu từ năm 1999, có thể không phản ánh hành vi người dùng hiện đại
    \item Chỉ xét các cặp 2 chuyên mục, chưa mở rộng cho n-itemsets với n > 2
    \item Chưa triển khai giao diện web để demo trực quan
\end{itemize}

\subsection{Hướng phát triển}

\begin{itemize}
    \item Tích hợp với hệ thống gợi ý real-time
    \item Mở rộng phân tích sequence (thứ tự click) bằng thuật toán GSP, PrefixSpan
    \item Kết hợp với các thuật toán Machine Learning khác (Collaborative Filtering)
    \item Xây dựng API và giao diện web demo
\end{itemize}

% ============================================================
% TÀI LIỆU THAM KHẢO
% ============================================================
\begin{thebibliography}{9}

\bibitem{zaki2000}
Zaki, M. J. (2000). 
\textit{Scalable Algorithms for Association Mining}. 
IEEE Transactions on Knowledge and Data Engineering, 12(3), 372-390.

\bibitem{han2011}
Han, J., Kamber, M., \& Pei, J. (2011). 
\textit{Data Mining: Concepts and Techniques} (3rd ed.). 
Morgan Kaufmann Publishers.

\bibitem{zaki2014}
Zaki, M. J., \& Meira Jr, W. (2014). 
\textit{Data Mining and Analysis: Fundamental Concepts and Algorithms}. 
Cambridge University Press.

\bibitem{agrawal1993}
Agrawal, R., Imielinski, T., \& Swami, A. (1993). 
\textit{Mining Association Rules Between Sets of Items in Large Databases}. 
Proceedings of the 1993 ACM SIGMOD International Conference on Management of Data.

\bibitem{uci_msnbc}
UCI Machine Learning Repository. 
\textit{MSNBC.com Anonymous Web Data Data Set}. 
\url{https://archive.ics.uci.edu/dataset/133/msnbc+com+anonymous+web+data}

\end{thebibliography}


\end{document}

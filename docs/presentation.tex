    \documentclass[10pt]{beamer}

    % Cấu hình theme
    \usetheme{Madrid}
    \usecolortheme{default}
    \usefonttheme{professionalfonts} % Sử dụng font chuyên nghiệp hơn
    \setbeamerfont{caption}{size=\scriptsize}

    % Gói hỗ trợ tiếng Việt và các gói khác
    \usepackage[utf8]{vietnam}
    \usepackage[utf8]{inputenc}
    \usepackage[T5]{fontenc}
    \usepackage{lmodern} % Font Latin Modern sắc nét hơn
    \usepackage[scaled]{helvet} % Dùng font Helvetica (giống Arial) làm font chính
    \renewcommand{\familydefault}{\sfdefault} % Chắc chắn dùng sans-serif

    \usepackage{bookmark}
    \usepackage{graphicx}
    \usepackage{booktabs}
    \usepackage{algorithm}
    \usepackage{algpseudocode}
    \usepackage{listings}
    \usepackage{xcolor}
    \usepackage{tikz}
    \usepackage{multirow}
    \usepackage{array}
    \usetikzlibrary{shapes,arrows,positioning,fit,calc}

    % Cấu hình code block đẹp hơn
    \usepackage{inconsolata} % Font code đẹp hơn (nếu có hỗ trợ) hoặc dùng basic tt
    \lstset{
        language=Python,
        basicstyle=\ttfamily\scriptsize, % Tăng size lên xíu cho dễ nhìn
        keywordstyle=\color{blue}\bfseries,
        commentstyle=\color{green!50!black}\itshape, % Comment in nghiêng, màu dịu hơn
        stringstyle=\color{orange!80!black},
        identifierstyle=\color{black},
        breaklines=true,
        frame=shadowbox, % Khung bóng đổ đẹp hơn
        rulesepcolor=\color{gray},
        showstringspaces=false,
        numbers=left,
        numberstyle=\tiny\color{gray},
        numbersep=5pt,
        tabsize=4,
        backgroundcolor=\color{yellow!5}, % Nền code màu nhạt dễ đọc
        escapeinside={(*@}{@*)}
    }

    % Định nghĩa lại tên thuật toán
    \floatname{algorithm}{Thuật toán}
    \renewcommand{\algorithmicrequire}{\textbf{Input:}}
    \renewcommand{\algorithmicensure}{\textbf{Output:}}

    % --- ĐỊNH NGHĨA TRANG BÌA CUSTOM ---
% --- ĐỊNH NGHĨA TRANG BÌA CUSTOM (MINIMALIST WHITE) ---
    \setbeamertemplate{title page}{
        \begin{tikzpicture}[remember picture,overlay]
            % Thanh màu xanh mỏng ở trên cùng
            \fill[structure.fg] (current page.north west) rectangle ([yshift=-0.2cm]current page.north east);
        \end{tikzpicture}
        
        \vspace{1.5cm}
        
        % Dòng tên trường
        {\tiny\bfseries\color{structure.fg!80} TRƯỜNG ĐẠI HỌC XÂY DỰNG HÀ NỘI \\ KHOA CÔNG NGHỆ THÔNG TIN \par}
        
        \vspace{0.5cm}
        
        % Tiêu đề chính (Font Serif to, đậm - Đã giảm size theo yêu cầu)
        {\usefont{T5}{put}{b}{n}\fontsize{16}{22}\selectfont\textcolor{black}{Ứng dụng thuật toán Eclat trong phân tích hành vi người dùng qua dữ liệu clickstream để gợi ý nội dung}\par}
        
        \vspace{0.3cm}
        
        % Subtitle
        {\color{gray} Báo cáo Bài tập lớn môn Khai phá dữ liệu \par}
        
        \vspace{1.5cm} 
        
        % Cột thông tin (Giảng viên & Sinh viên)
        \begin{columns}[t]
            \column{0.45\textwidth}
                {\scriptsize\bfseries\color{structure.fg} GIẢNG VIÊN HƯỚNG DẪN}\\[0.1cm]
                {\large\bfseries TS. Phạm Hồng Phong}
                
            \column{0.55\textwidth}
                {\scriptsize\bfseries\color{structure.fg} NHÓM THỰC HIỆN}\\[0.1cm]
                {\small
                1. Nguyễn Việt Anh (0203968)\\
                2. Nguyễn Việt Hùng (0208768)\\
                3. Đỗ Quang Hợp (0208568)
                }
        \end{columns}
        
        \vfill
    }

    % Thông tin Metadata
    \title[Thuật toán Eclat]{Ứng dụng thuật toán Eclat trong phân tích hành vi\\người dùng qua dữ liệu clickstream để gợi ý nội dung}

    \author[Việt Anh - Việt Hùng - Quang Hợp]{
        Nguyễn Việt Anh \and Nguyễn Việt Hùng \and Đỗ Quang Hợp
    }
    \institute[ĐH Xây Dựng HN]{
        Khoa Công nghệ Thông tin \\
        Trường Đại học Xây Dựng Hà Nội
    }
    \date{Tháng 2 năm 2026}

    \begin{document}

    % Slide 1: Trang bìa
    \begin{frame}
        \titlepage
    \end{frame}

    % Slide 2: Mục lục
    \begin{frame}{Nội dung trình bày}
        \tableofcontents
    \end{frame}

    % ============================================================
    % 1. TỔNG QUAN
    % ============================================================
    \section{Tổng quan đề tài}

    \begin{frame}{Đặt vấn đề}
        \textbf{Bối cảnh thực tế:}
        \begin{itemize}
            \item Sự bùng nổ của Internet $\rightarrow$ Website/Ứng dụng tạo ra lượng dữ liệu \textbf{clickstream} khổng lồ.
            \item Clickstream ghi lại toàn bộ lịch sử thao tác: Trang đã xem, thời gian xem, thứ tự click...
        \end{itemize}
        
        \vspace{0.3cm}
        \textbf{Vấn đề cần giải quyết:}
        \begin{itemize}
            \item Làm sao hiểu được hành vi và sở thích người dùng từ "đống" log hỗn độn?
            \item Làm sao để biết: "Người đọc tin Thể thao thường quan tâm tin gì tiếp theo?"
        \end{itemize}
        
        \vspace{0.3cm}
        \textbf{Giải pháp:}
        \begin{itemize}
            \item Sử dụng \textbf{Khai phá luật kết hợp (Association Rule Mining)}.
            \item Kỹ thuật này giúp tìm ra các mối quan hệ ẩn: $X \Rightarrow Y$.
        \end{itemize}
    \end{frame}

    \begin{frame}{Mục tiêu đề tài}
        Đề tài tập trung vào 4 mục tiêu chính:
        \begin{enumerate}
            \item \textbf{Nghiên cứu thuật toán}: Tìm hiểu sâu và triển khai thuật toán \textbf{Eclat} từ đầu (from scratch), không dùng thư viện có sẵn.
            \item \textbf{Xử lý dữ liệu thực}: Áp dụng thuật toán trên bộ dữ liệu \textbf{MSNBC.com Anonymous Web Data} (gần 1 triệu giao dịch).
            \item \textbf{Sinh luật gợi ý}: Tạo ra các luật dạng "Nếu xem A $\rightarrow$ Gợi ý B".
            \item \textbf{Đánh giá}: Phân tích hiệu quả các luật dựa trên các chỉ số Support, Confidence, Lift.
        \end{enumerate}
    \end{frame}

    \begin{frame}{Phạm vi nghiên cứu}
        \begin{itemize}
            \item \textbf{Dữ liệu đầu vào}: 
            \begin{itemize}
                \item Nguồn: UCI Machine Learning Repository.
                \item Dữ liệu: Clickstream từ trang tin tức MSNBC.com.
                \item Kích thước: 989,818 phiên truy cập.
            \end{itemize}
            
            \item \textbf{Thuật toán}: 
            \begin{itemize}
                \item Tập trung duy nhất vào \textbf{Eclat (Equivalence Class Transformation)}.
                \item So sánh lý thuyết với Apriori.
            \end{itemize}
            
            \item \textbf{Công nghệ}:
            \begin{itemize}
                \item Ngôn ngữ: Python 3.x.
                \item Cách tiếp cận: Tự cài đặt cấu trúc dữ liệu và giải thuật.
            \end{itemize}
        \end{itemize}
    \end{frame}

    % ============================================================
    % 2. CƠ SỞ LÝ THUYẾT
    % ============================================================
    \section{Cơ sở lý thuyết}

    \begin{frame}{1. Các khái niệm cơ bản (1)}
        \begin{block}{Tập mục (Itemset)}
            Cho $I = \{i_1, i_2, ..., i_n\}$ là tập hợp các mặt hàng.
            \begin{itemize}
                \item \textbf{Itemset} $X$ là một tập con của $I$ ($X \subseteq I$).
                \item \textbf{k-itemset}: Itemset có kích thước $k$.
                \item \textit{Ví dụ:} $I = \{A, B, C, D\}$, $X = \{A, C\}$ là 2-itemset.
            \end{itemize}
        \end{block}

        \begin{block}{Cơ sở dữ liệu giao dịch (Transaction Database)}
            CSDL $D$ gồm các giao dịch $T = \{T_1, T_2, ..., T_m\}$.
            \begin{itemize}
                \item Mỗi giao dịch $T_j \subseteq I$ có một định danh duy nhất (TID).
                \item $T_j$ chứa các mục được mua cùng nhau trong một thời điểm.
            \end{itemize}
        \end{block}
    \end{frame}

    \begin{frame}{2. Các khái niệm cơ bản (2)}
        \begin{alertblock}{Độ hỗ trợ (Support)}
            Tỷ lệ (xác suất) xuất hiện của itemset $X$ trong CSDL $D$.
            \[ Support(X) = \frac{|\{T \in D \mid X \subseteq T\}|}{|D|} \in [0, 1] \]
        \end{alertblock}

        \begin{exampleblock}{Tập phổ biến (Frequent Itemset)}
            Itemset $X$ được gọi là \textbf{phổ biến} nếu độ hỗ trợ của nó không nhỏ hơn một ngưỡng tối thiểu do người dùng quy định ($min\_support$).
            \[ Support(X) \geq min\_support \]
        \end{exampleblock}
        
        \vspace{0.5cm}
        \textbf{Mục tiêu của thuật toán}: Tìm tất cả các tập phổ biến trong $D$.
    \end{frame}

    \begin{frame}{3. Các khái niệm cơ bản (3)}
        \begin{block}{Luật kết hợp (Association Rule)}
            Có dạng $X \Rightarrow Y$, trong đó $X, Y \subseteq I$ và $X \cap Y = \emptyset$.
            \\ \textit{Ý nghĩa:} "Nếu mua $X$ thì có khả năng sẽ mua thêm $Y$".
        \end{block}

        \begin{columns}[t]
            \column{0.48\textwidth}
                \begin{block}{Độ tin cậy (Confidence)}
                    Độ chắc chắn của luật:
                    \[ Conf(X \Rightarrow Y) = \frac{Sup(X \cup Y)}{Sup(X)} \]
                    (Xác suất có điều kiện $P(Y|X)$)
                \end{block}
            
            \column{0.48\textwidth}
                \begin{block}{Độ tương quan (Lift)}
                    Độ tương quan giữa X và Y:
                    \[ Lift(X \Rightarrow Y) = \frac{Conf(X \Rightarrow Y)}{Sup(Y)} \]
                    \begin{itemize}
                        \item $>1$: Tương quan dương (Tốt)
                        \item $=1$: Độc lập (Không tốt)
                        \item $<1$: Tương quan âm
                    \end{itemize}
                \end{block}
        \end{columns}
    \end{frame}

    \begin{frame}{Ví dụ minh họa (Theo Báo cáo)}
        \textbf{CSDL Giao dịch ($|D|=5$):}
        \begin{table}
        \centering\small
        \begin{tabular}{|c|l|}
        \hline \textbf{TID} & \textbf{Items} \\ \hline
        T1 & Bánh mì, Sữa, Bơ \\
        T2 & Bánh mì, Sữa \\
        T3 & Sữa, Bơ \\
        T4 & Bánh mì, Bơ \\
        T5 & Bánh mì, Sữa, Bơ \\ \hline
        \end{tabular}
        \end{table}

        \textbf{Xét luật: \{Bánh mì\} $\Rightarrow$ \{Sữa\}} ("Mua Bánh mì thì mua Sữa")
        \begin{columns}[t]
            \column{0.5\textwidth}
                \begin{itemize}
                    \item $Sup(\text{Bánh mì}) = 4/5 = 80\%$
                    \item $Sup(\text{Sữa}) = 4/5 = 80\%$
                    \item $Sup(\text{Bánh mì} \cup \text{Sữa}) = 3/5 = 60\%$
                \end{itemize}
            
            \column{0.5\textwidth}
                \textbf{Đánh giá:}
                \begin{itemize}
                    \item \textbf{Confidence}: $\frac{60\%}{80\%} = 75\%$ (Khá cao)
                    \item \textbf{Lift}: $\frac{60\%}{80\% \times 80\%} = 0.9375 < 1$
                \end{itemize}
        \end{columns}
        \vspace{0.2cm}
        \textit{Kết luận:} Dù Confidence cao nhưng Lift < 1 $\rightarrow$ Tương quan âm (không tốt để gợi ý).
    \end{frame}

    \begin{frame}{Thuật toán Eclat: Giới thiệu}
        \begin{alertblock}{Định nghĩa}
            \textbf{Eclat} (\textbf{E}quivalence \textbf{C}lass \textbf{C}lustering and bottom-up \textbf{La}ttice \textbf{T}raversal)
            \begin{itemize}
                \item Đề xuất bởi: \textbf{Mohammed J. Zaki} (2000).
                \item Là thuật toán khai phá tập phổ biến dựa trên hướng tiếp cận \textbf{DFS (Depth-First Search)}.
            \end{itemize}
        \end{alertblock}
        
        \vspace{0.5cm}
        \textbf{Ưu điểm vượt trội so với Apriori:}
        \begin{enumerate}
            \item \textbf{Chỉ quét CSDL một lần duy nhất}: Để chuyển đổi dữ liệu sang định dạng dọc (Vertical Format).
            \item \textbf{Không cần sinh ứng viên (No Candidate Generation)}: Tránh việc tạo ra số lượng lớn itemset thừa.
            \item \textbf{Tính toán Support nhanh}: Sử dụng phép giao tập hợp (Intersection) các TID-Set (nhanh hơn nhiều so với việc đếm pattern matching).
        \end{enumerate}
    \end{frame}

    \begin{frame}{Horizontal vs Vertical Format}
        \begin{columns}
            \begin{column}{0.48\textwidth}
                \textbf{Horizontal (Truyền thống)}
                \begin{table}
                \centering\scriptsize
                \begin{tabular}{|c|l|}
                \hline TID & Items \\ \hline
                1 & A, B, D \\
                2 & B, C \\
                3 & A, B, C \\
                \hline
                \end{tabular}
                \end{table}
                \begin{itemize}
                    \item Dễ lưu trữ tự nhiên.
                    \item Tính support phải scan toàn bộ.
                \end{itemize}
            \end{column}
            \begin{column}{0.48\textwidth}
                \textbf{Vertical (Eclat)}
                \begin{table}
                \centering\scriptsize
                \begin{tabular}{|c|l|}
                \hline Item & TID-Set \\ \hline
                A & \{1, 3\} \\
                B & \{1, 2, 3\} \\
                C & \{2, 3\} \\
                D & \{1\} \\
                \hline
                \end{tabular}
                \end{table}
                \begin{itemize}
                    \item Map: Item $\rightarrow$ List of TIDs.
                    \item Support = Size of TID-Set.
                \end{itemize}
            \end{column}
        \end{columns}
    \end{frame}

    \begin{frame}{Nguyên lý hoạt động của Eclat}
        \textbf{Nguyên lý cốt lõi:}
        \[ TID(X \cup Y) = TID(X) \cap TID(Y) \]
        
        \vspace{0.3cm}
        \textbf{Ý nghĩa:}
        \begin{itemize}
            \item Danh sách giao dịch chứa cả $X$ và $Y$ chính là giao của danh sách chứa $X$ và danh sách chứa $Y$.
            \item Support của tập mới = Kích thước tập giao này.
        \end{itemize}
        
        \vspace{0.3cm}
        \textbf{Quy trình đệ quy (DFS):}
        \begin{itemize}
            \item Từ các items đơn lẻ (1-itemsets), giao chúng để tạo 2-itemsets.
            \item Từ 2-itemsets, giao tiếp để tạo 3-itemsets...
            \item Dừng khi tập giao rỗng hoặc không đủ min\_support.
        \end{itemize}
    \end{frame}

    \begin{frame}[fragile]{Mã giả thuật toán Eclat}
        \begin{algorithmic}[1]
            \Procedure{Eclat\_Recursive}{$Prefix, TID\_Subset, min\_sup$}
                \While{$TID\_Subset \neq \emptyset$}
                    \State $(item, tids) \gets POP(TID\_Subset)$
                    \State $NewItemset \gets Prefix \cup \{item\}$
                    \State \textbf{Output} $(NewItemset, |tids|)$
                    
                    \State $NewSubset \gets \emptyset$
                    \ForAll{$(other, other\_tids) \in TID\_Subset$}
                        \State $Intersect \gets tids \cap other\_tids$ \Comment{Phép giao}
                        \If{$|Intersect| \geq min\_sup$}
                            \State $ADD(NewSubset, (other, Intersect))$
                        \EndIf
                    \EndFor
                    
                    \If{$NewSubset \neq \emptyset$}
                        \State \Call{Eclat\_Recursive}{$NewItemset, NewSubset, min\_sup$}
                    \EndIf
                \EndWhile
            \EndProcedure
        \end{algorithmic}
    \end{frame}

    \begin{frame}{Ví dụ chạy tay Eclat: Bước 1 (Input)}
        \textbf{Input DB:}
        \begin{itemize}
            \item T1: A, B, C
            \item T2: A, B
            \item T3: A, C
        \end{itemize}
        \textbf{Min Support = 2}
        
        \vspace{0.5cm}
        \textbf{Chuyển đổi sang Vertical:}
        \begin{itemize}
            \item A: \{1, 2, 3\} (Supp=3) $\rightarrow$ Giữ
            \item B: \{1, 2\} (Supp=2) $\rightarrow$ Giữ
            \item C: \{1, 3\} (Supp=2) $\rightarrow$ Giữ
        \end{itemize}
    \end{frame}

    \begin{frame}{Ví dụ chạy tay Eclat: Bước 2 (Giao 2-itemsets)}
        Duyệt danh sách items $\{A, B, C\}$:
        
        \vspace{0.3cm}
        \textbf{1. Xét A (với B, C):}
        \begin{itemize}
            \item $TID(A \cap B) = \{1, 2, 3\} \cap \{1, 2\} = \{1, 2\}$. Count = 2 $\ge$ 2.
            \item $\rightarrow$ \textbf{\{A, B\} là tập phổ biến}.
            \item $TID(A \cap C) = \{1, 2, 3\} \cap \{1, 3\} = \{1, 3\}$. Count = 2 $\ge$ 2.
            \item $\rightarrow$ \textbf{\{A, C\} là tập phổ biến}.
        \end{itemize}
        
        \vspace{0.3cm}
        \textbf{2. Xét B (với C):} (Sau khi đã xét A xong)
        \begin{itemize}
            \item $TID(B \cap C) = \{1, 2\} \cap \{1, 3\} = \{1\}$. Count = 1 < 2.
            \item $\rightarrow$ Loại.
        \end{itemize}
    \end{frame}

    \begin{frame}{Ví dụ chạy tay Eclat: Bước 3 (3-itemsets)}
        \textbf{Đệ quy tiếp từ nhánh của A:}
        \begin{itemize}
            \item Ta có tập con sinh ra từ A là: $\{(B, \{1,2\}), (C, \{1,3\})\}$
            \item Giao tiếp B và C trong nhánh này:
            \item $TID(A \cap B \cap C) = \{1, 2\} \cap \{1, 3\} = \{1\}$.
            \item Count = 1 < 2 $\rightarrow$ Dừng.
        \end{itemize}
        
        \vspace{0.5cm}
        \textbf{Kết quả cuối cùng:}
        \begin{itemize}
            \item \{A\}, \{B\}, \{C\}
            \item \{A, B\}, \{A, C\}
        \end{itemize}
    \end{frame}

    \begin{frame}{So sánh Apriori và Eclat}
        \begin{table}
        \centering\small
        \begin{tabular}{|l|p{4cm}|p{4cm}|}
            \hline
            \textbf{Tiêu chí} & \textbf{Apriori} & \textbf{Eclat} \\
            \hline
            \textbf{Cấu trúc} & Horizontal & \textbf{Vertical} \\
            \hline
            \textbf{Quét DB} & Nhiều lần (k lần) & \textbf{1 lần} (Tiền xử lý) \\
            \hline
            \textbf{Duyệt} & BFS (Chiều rộng) & \textbf{DFS (Chiều sâu)} \\
            \hline
            \textbf{Tính Support} & Đếm pattern matching & \textbf{Phép giao TID} (Rất nhanh) \\
            \hline
            \textbf{Bộ nhớ} & Thấp & \textbf{Cao hơn} (Lưu TID lists) \\
            \hline
            \textbf{Hiệu năng} & Chậm với dữ liệu lớn & \textbf{Nhanh hơn đáng kể} \\
            \hline
        \end{tabular}
        \end{table}
    \end{frame}

    % ============================================================
    % 3. DỮ LIỆU & TIỀN XỬ LÝ
    % ============================================================
    \section{Dữ liệu và Tiền xử lý}

    \begin{frame}{Giới thiệu bộ dữ liệu MSNBC}
        \textbf{Nguồn gốc:}
        \begin{itemize}
            \item UCI Machine Learning Repository (1999).
            \item Log truy cập của người dùng trên trang tin tức MSNBC.com.
        \end{itemize}
        
        \vspace{0.3cm}
        \textbf{Thống kê:}
        \begin{itemize}
            \item Tổng số phiên (sessions): \textbf{989,818} (Gần 1 triệu).
            \item Số chuyên mục (items): \textbf{17}.
            \item Trung bình click/phiên: 5.7.
        \end{itemize}
    \end{frame}

    \begin{frame}{Cấu trúc và Ánh xạ dữ liệu}
        \textbf{File gốc (.seq):}
        \begin{itemize}
            \item Mỗi dòng là một danh sách các số nguyên.
            \item Ví dụ: \texttt{1 1 5 1 2} (Người dùng xem: Frontpage $\rightarrow$ Frontpage $\rightarrow$ Opinion $\rightarrow$ Frontpage $\rightarrow$ News).
        \end{itemize}

        \vspace{0.3cm}
        \textbf{Bảng ánh xạ (Mapping):}
        \begin{table}
        \centering\tiny
        \begin{tabular}{|c|l|c|l|}
        \hline ID & Tên mục & ID & Tên mục \\ \hline
        1 & Frontpage (Trang chủ) & 10 & Living (Đời sống)\\
        2 & News (Tin tức) & 11 & Business (Kinh doanh)\\
        3 & Tech (Công nghệ) & 12 & Sports (Thể thao)\\
        4 & Local (Địa phương) & 13 & Summary (Tóm tắt)\\
        5 & Opinion (Ý kiến) & 14 & BBS (Diễn đàn)\\
        6 & On-air (Phát sóng) & ... & ...\\
        \hline
        \end{tabular}
        \end{table}
    \end{frame}

    \begin{frame}{Quy trình tiền xử lý}
        \begin{enumerate}
            \item \textbf{Load Data}: Đọc file từng dòng, bỏ qua metadata (các dòng bắt đầu bằng \%).
            \item \textbf{Split}: Tách chuỗi số thành list các ID.
            \item \textbf{Mapping}: Chuyển đổi ID số $\rightarrow$ Tên chuyên mục (String).
            \item \textbf{Deduplicate}: Sử dụng \texttt{set()} để loại bỏ các lần xem trùng lặp trong cùng 1 phiên. 
            \begin{itemize}
                \item Lý do: Eclat quan tâm "CÓ xem hay KHÔNG", không quan tâm xem bao nhiêu lần.
            \end{itemize}
            \item \textbf{Filter}: Loại bỏ các phiên rỗng.
        \end{enumerate}
    \end{frame}

    \begin{frame}[fragile]{Code tiền xử lý (Python)}
        \begin{lstlisting}
        def load_data(filepath):
            dataset = []
            with open(filepath, 'r') as f:
                for line in f:
                    # Bo qua metadata
                    if not line or line.startswith('%'):
                        continue
                    
                    item_ids = line.strip().split()
                    
                    # Loai bo trung lap bang set
                    transaction = set()
                    for item_id in item_ids:
                        if item_id in CATEGORY_NAMES:
                            transaction.add(CATEGORY_NAMES[item_id])
                    
                    if transaction:
                        dataset.append(transaction)
            return dataset
        \end{lstlisting}
    \end{frame}

    % ============================================================
    % 4. TRIỂN KHAI
    % ============================================================
    \section{Triển khai thuật toán}

    \begin{frame}{Kiến trúc chương trình}
        Chương trình được chia thành 3 modules chính:
        
        \vspace{0.5cm}
        \centering
        \begin{tikzpicture}[
            node distance=0.8cm,
            box/.style={rectangle, draw, fill=blue!10, minimum width=3cm, minimum height=1cm, align=center, rounded corners},
            arrow/.style={->, thick, >=stealth}
        ]
            \node[box] (main) {main.py\\(Điều phối)};
            \node[box, below left=of main] (data) {data\_loader.py\\(Xử lý dữ liệu)};
            \node[box, below=of main] (algo) {eclat\_algo.py\\(Lõi thuật toán)};
            \node[box, below right=of main] (utils) {utils.py\\(Sinh luật)};
            
            \draw[arrow] (main) -- (data);
            \draw[arrow] (main) -- (algo);
            \draw[arrow] (main) -- (utils);
            \draw[arrow] (data) -- (algo) node[midway, below, font=\tiny] {Dataset};
            \draw[arrow] (algo) -- (utils) node[midway, below, font=\tiny] {Itemsets};
        \end{tikzpicture}
    \end{frame}

    \begin{frame}[fragile]{Module Eclat: Khởi tạo \& Fit}
        \begin{lstlisting}
    class Eclat:
        def __init__(self, min_support=0.01, min_items=1):
            self.min_support = min_support
            self.frequent_itemsets = []

        def fit(self, dataset):
            self._total_trans = len(dataset)
            min_supp_count = self._total_trans * self.min_support
            
            # 1. Chuyen sang Vertical Format
            tid_dict = {}
            for tid, trans in enumerate(dataset):
                for item in trans:
                    if item not in tid_dict: tid_dict[item] = set()
                    tid_dict[item].add(tid)
            
            # 2. Loc & Sap xep
            tid_dict = {k: v for k, v in tid_dict.items() 
                        if len(v) >= min_supp_count}
            sorted_items = sorted(tid_dict.items(), 
                                key=lambda x: len(x[1]), reverse=True)
                                
            # 3. Goi de quy
            self._eclat_recursive([], sorted_items, min_supp_count)
        \end{lstlisting}
    \end{frame}

    \begin{frame}[fragile]{Module Eclat: Hàm Đệ quy (Core)}
        \begin{lstlisting}
        def _eclat_recursive(self, prefix, tid_subset, min_supp_count):
            while tid_subset:
                item, tids = tid_subset.pop(0)
                new_itemset = prefix + [item]
                
                # Luu ket qua
                self.frequent_itemsets.append((new_itemset, len(tids)))
                
                # Tao nhanh moi
                new_tid_subset = []
                for other_item, other_tids in tid_subset:
                    # PHEP GIAO QUAN TRONG NHAT
                    intersect = tids & other_tids
                    
                    if len(intersect) >= min_supp_count:
                        new_tid_subset.append((other_item, intersect))
                
                # De quy
                if new_tid_subset:
                    self._eclat_recursive(new_itemset, new_tid_subset, min_supp_count)
        \end{lstlisting}
    \end{frame}

    \begin{frame}[fragile]{Module Utils: Sinh luật gợi ý}
        Logic sinh luật từ tập phổ biến:
        \begin{itemize}
            \item Duyệt qua các frequent itemsets có kích thước $\ge 2$.
            \item Với mỗi itemset $I$, tạo luật $A \Rightarrow B$ (với $A \subset I, B = I \setminus A$).
            \item Tính Confidence và Lift. Nếu đạt ngưỡng $\rightarrow$ Lưu.
        \end{itemize}

        \begin{lstlisting}
        # Tinh Confidence
        supp_AB = count_AB / total
        supp_A = lookup[A] / total
        confidence = supp_AB / supp_A
        
        # Tinh Lift
        supp_B = lookup[B] / total
        lift = supp_AB / (supp_A * supp_B)
        \end{lstlisting}
    \end{frame}

    % ============================================================
    % 5. KẾT QUẢ
    % ============================================================
    \section{Kết quả thực nghiệm}

    \begin{frame}{Cấu hình thực nghiệm}
        Chúng tôi chạy thực nghiệm trên toàn bộ dữ liệu MSNBC với cấu hình:
        
        \begin{alertblock}{Tham số}
            \begin{itemize}
                \item \textbf{Min Support}: $2\%$ (0.02)
                \begin{itemize}
                    \item Tương đương: Itemset phải xuất hiện trong ít nhất $\approx 20,000$ phiên.
                \end{itemize}
                \item \textbf{Min Confidence}: $40\%$ (0.4)
                \begin{itemize}
                    \item Tương đương: Nếu xem A, phải có ít nhất 40\% khả năng xem B.
                \end{itemize}
            \end{itemize}
        \end{alertblock}
    \end{frame}

    \begin{frame}{Kết quả tổng quan}
        \textbf{Tập phổ biến (Frequent Itemsets):}
        \begin{itemize}
            \item Tìm thấy tổng cộng: \textbf{32} itemsets phổ biến.
            \item 15/17 chuyên mục thỏa mãn min\_support (1-itemsets).
            \item 17 cặp chuyên mục thỏa mãn min\_support (2-itemsets).
        \end{itemize}

        \vspace{0.3cm}
        \textbf{Top 3 Chuyên mục được xem nhiều nhất:}
        \begin{enumerate}
            \item \textbf{Frontpage} (Trang chủ): 940k lượt (95\% sessions).
            \item \textbf{News} (Tin tức): 452k lượt.
            \item \textbf{Weather} (Thời tiết): 439k lượt.
        \end{enumerate}
    \end{frame}

    \begin{frame}{Các luật gợi ý tiêu biểu (Strongest Rules)}
        Bảng dưới đây hiển thị các luật có độ tin cậy và tương quan cao nhất:

        \begin{table}
        \centering
        \small
        \begin{tabular}{|l|l|c|c|l|}
        \hline
        \textbf{Antecedent} & \textbf{Consequent} & \textbf{Conf.} & \textbf{Lift} & \textbf{Đánh giá} \\
        \hline
        Business & Frontpage & \textbf{56.8\%} & 1.80 & Rất mạnh \\
        Local & Frontpage & 54.2\% & 1.72 & Mạnh \\
        Living & Frontpage & 51.9\% & 1.64 & Khá \\
        Misc & On-air & 41.3\% & \textbf{1.88} & Tương quan cao \\
        Opinion & Frontpage & 48.1\% & 1.52 & Khá \\
        \hline
        \end{tabular}
        \end{table}
        
        \textit{*Note: Misc = Tổng hợp, On-air = Phát sóng video}
    \end{frame}

    \begin{frame}{Phân tích chi tiết}
        \textbf{1. Mối quan hệ Business $\rightarrow$ Frontpage (Conf=56.8\%, Lift=1.80)}
        \begin{itemize}
            \item \textbf{Insight}: Người đọc tin Kinh doanh có xu hướng rất cao quay lại Trang chủ.
            \item \textbf{Hành vi}: Họ thường là người cập nhật tin tức tổng quát, sau khi đọc tin thị trường sẽ quay ra xem tin nóng khác.
            \item \textbf{Hành động}: Đặt nút "Về trang chủ" to rõ hoặc Widget "Tin nóng trang chủ" ngay trong bài viết Kinh doanh.
        \end{itemize}

        \vspace{0.3cm}
        \textbf{2. Mối quan hệ Misc $\rightarrow$ On-air (Lift=1.88 - Cao nhất)}
        \begin{itemize}
            \item \textbf{Insight}: Dù Confidence (41\%) không cao nhất, nhưng Lift cao nhất cho thấy hai mục này hút nhau mạnh mẽ. Người xem tin Tổng hợp rất thích xem Video.
            \item \textbf{Hành động}: Nhúng video (On-air) vào các tin Tổng hợp (Misc) để tăng view.
        \end{itemize}
    \end{frame}

    % ============================================================
    % 6. KẾT LUẬN
    % ============================================================
    \section{Kết luận}

    \begin{frame}{Đóng góp của đề tài}
        \begin{enumerate}
            \item \textbf{Hiểu sâu thuật toán}: Đã tự tay cài đặt Eclat, hiểu rõ cơ chế Vertical Format và DFS.
            \item \textbf{Xử lý Big Data}: Đã xử lý thành công file dữ liệu gần 1 triệu dòng mà không bị tràn bộ nhớ (nhờ tối ưu hóa Set).
            \item \textbf{Giá trị thực tiễn}: Các luật tìm ra hoàn toàn hợp lý và có thể ứng dụng để:
            \begin{itemize}
                \item Sắp xếp lại menu website.
                \item Xây dựng hệ thống Recommender System cơ bản.
            \end{itemize}
        \end{enumerate}
    \end{frame}

    \begin{frame}{Hạn chế và Hướng phát triển}
        \textbf{Hạn chế:}
        \begin{itemize}
            \item Dữ liệu cũ (năm 1999) có thể không phản ánh hành vi người dùng hiện đại (thích video ngắn, mxh...).
            \item Thuật toán chưa tối ưu cho việc cập nhật dữ liệu mới (Incremental Mining).
        \end{itemize}
        
        \vspace{0.5cm}
        \textbf{Hướng phát triển:}
        \begin{itemize}
            \item \textbf{Sequence Mining}: Sử dụng thuật toán GSP hoặc PrefixSpan để xét đến \textit{thứ tự} thời gian (Ví dụ: Xem A rồi mới xem B khác với xem B rồi mới xem A).
            \item \textbf{Giao diện Demo}: Xây dựng Web App trực quan hóa các luật kết hợp dạng đồ thị (Graph).
        \end{itemize}
    \end{frame}

    \begin{frame}
        \centering
        \Huge \textbf{CẢM ƠN THẦY VÀ CÁC BẠN\\ĐÃ LẮNG NGHE!}
        
        \vspace{1cm}
        \large \textbf{Q \& A}
    \end{frame}

    \end{document}
